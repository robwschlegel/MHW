\documentclass[a4paper,10pt,review]{elsarticle}

\usepackage{lineno,hyperref}
\modulolinenumbers[5]
\frenchspacing

% Inserted by AJS
\usepackage{upquote}
\usepackage{textgreek}
\usepackage{microtype} % place after fonts; even better typesetting for improved readability
\usepackage{xfrac} % nice fractions
\usepackage[color=yellow, textsize=tiny]{todonotes}
\usepackage{gensymb}
\usepackage{amsmath, amssymb}

\journal{Progress in Oceanography}

%%%%%%%%%%%%%%%%%%%%%%%
%% Elsevier bibliography styles
%%%%%%%%%%%%%%%%%%%%%%%
%% To change the style, put a % in front of the second line of the current style and
%% remove the % from the second line of the style you would like to use.
%%%%%%%%%%%%%%%%%%%%%%%

%% Numbered
%\bibliographystyle{model1-num-names}

%% Numbered without titles
%\bibliographystyle{model1a-num-names}

%% Harvard
%\bibliographystyle{model2-names.bst}\biboptions{authoryear}

%% Vancouver numbered
%\usepackage{numcompress}\bibliographystyle{model3-num-names}

%% Vancouver name/year
%\usepackage{numcompress}\bibliographystyle{model4-names}\biboptions{authoryear}

%% APA style
%\bibliographystyle{model5-names}\biboptions{authoryear}

%% AMA style
%\usepackage{numcompress}\bibliographystyle{model6-num-names}

%% `Elsevier LaTeX' style
\bibliographystyle{elsarticle-num}
%%%%%%%%%%%%%%%%%%%%%%%

\begin{document}

\begin{frontmatter}

\title{The co-occurence of marine heat waves and cold spells in nearshore and offshore regions along South Africa}

%% or include affiliations in footnotes:
\author[firstaddress]{Robert W. Schlegel\corref{mycorrespondingauthor}}
\cortext[mycorrespondingauthor]{Corresponding author}
\ead{wiederweiter@gmail.com}

\author[secondaddress,thirdaddress]{Eric C. J. Oliver}
\author[fourthaddress]{Thomas W. Wernberg}
\author[firstaddress]{Albertus J. Smit}

% \author[mysecondaryaddress]{Global Customer Service\corref{mycorrespondingauthor}}

\address[firstaddress]{Department for Biodiversity and Conservation Biology, University of the Western Cape, Private Bag X17, Bellville 7535, South Africa}

\address[secondaddress]{ARC Centre of Excellence for Climate System Science, The University of New South Wales, Sydney, Australia}

\address[thirdaddress]{Institute for Marine and Antarctic Studies, University of Tasmania, Hobart, Australia}

\address[fourthaddress]{UWA Oceans Institute and School of Plant Biology, The University of Western Australia, Crawley, 6009 Western Australia, Australia}

\begin{abstract}
The term marine heat wave (MHW) was first coined in 2013 with no central definition having been agreed upon before. This lack of a definition had led to an inability of different research groups to compare their findings on this phenomenon before 2013. In order to assuage this issue, a research team has recently created a definition for MHWs that will be valid anywhere in the world. We have taken this algorithm and applied it to the \emph{in situ} time series available for the coast of South Africa that are longer than 10 years and with at least 90\% complete daily records. It was also decided to apply the algorithm to cool cold temperatures and investigate the presence of marine cold spells (MCSs). We found that MHWs and MCSs can be found along the entire stretch of South Africa's coastline and with some temporal and spatial agreement between the largest events detected. MHWs occur more often, last longer than MCSs and have greater cumulative intensities. There was little variance in the cumulative intensity [\degree C \texttimes~days] around the mean for MHWs and MCSs however, several were much larger and there tended to be specific time series that displayed more dramatic results than others. The coastline was further divided into three sections (west, south, and east) to investigate the effect of geography on MHWs and MCSs and it was found that the south coast experiences more, longer and more intense MHWs and MCSs than the other two coastlines. The mechanism driving the higher intensity of events on the south coast, which is much greater than the other coasts, requires further study. The largest three MHWs of most time series along the coast of South Africa have occurred in the second half of the time series whereas the largest three MCSs have occurred in the first half. These same calculations were conducted for offshore temperatures from NOAA optimally interpolated sea surface temperature (OISST) data, too. It was found that the proportion of co-occurrence between \emph{in situ} and OISST data ranged from 0.5--0.0 for each coastline with co-occurrence rates being the largest on the south coast. Few time series showed co-occurrence amongst the 50\% largest events.
\end{abstract}

\begin{keyword}
marine heat waves \sep marine cold spells \sep OISST \sep \emph{in situ} data \sep co-occurence
\end{keyword}

\end{frontmatter}

\linenumbers

\section{Introduction}

Over the past three decades, global-scale anthropogenically mediated warming has negatively affected marine and terrestrial realms with far reaching consequences for humanity and natural ecological functioning. \todo{Is this the best way we want to start framing this work? Given that we can’t really say much about the long-term trends given that most stations are in the 10-20 years range of data record length.} Although climate change is generally understood as a gradual long-term rise in global mean surface temperature (IPCC 2012), which will continue for decades or centuries, it is generally the associated increase in frequency and severity of extreme events that affects humans and ecosystems alike in the short-term (Easterling et al. 2000). Impacts are often sudden with catastrophic consequences. Such extreme events include droughts, floods, wind storms, tropical cyclones, heat waves and cold spells. `Pulse' events exceeding certain thresholds of frequency, intensity (extremeness), duration, timing and rate of onset (abruptness) can drive punctuated perturbations to species distributions, which eventually modify the structure and function of ecosystems (Wernberg et al. 2013; Rehage et al. 2016), and the recognition to focus more on events and less on trends has emerged as a recent direction of climate change research (Jentsch et al. 2007). \todo{This is good and important, but extreme events happen regardless of climate change. This paper mostly focusses on the mean state not the changes. Is this not better suited to the discussion when you need to start talking about impacts/implications?}

The focus of this paper is on periods of consecutive days when seawater temperature is statistically extreme with respect to normal, thereby including seasonally anomalous warm (or cold) events. The concept of heat waves is usually applied to atmospheric phenomena where vague definitions such as ``a period of abnormally and uncomfortably hot and usually humid weather'' are invoked (Glickman 2000), but there are also examples of more precise definitions that rely on statistical properties and other metrics of the temperature record that are relative to location and time of year (e.g. Meehl and Tebaldi 2004; Alexander et al. 2006; Fischer and Shär 2010; Fischer et al. 2011). Recent years have seen investigations of heat waves in the ocean due to them becoming more frequent over time (e.g. MacKenzie and Schiedek 2007; Selig et al. 2010; Sura 2011; Lima and Wethey 2012; deCastro 2014). Well documented marine heat waves (MHW) have occurred in the Mediterranean in 2003 (e.g. Black et al. 2004; Olita et al. 2007; Garrabou et al. 2009), off the coast of Western Australia in 2011 (e.g. Feng et al. 2013; Pearce and Feng 2013; Wernberg et al. 2013), in the north west Atlantic Ocean in 2012 (e.g. Mills et al. 2013; Chen et al. 2014; Chen et al. 2015) and now the ``Blob'' from 2014 to 2016 in the north east Pacific Ocean (Bond et al. 2015). The extreme temperatures from these events, and others like them, may have wide ranging negative impacts upon the local ecology for the regions in which they occur. For example, the 2003 Mediterranean heat wave may have affected up to 80\% of the gorgonian fan colonies in certain areas of this sea (Garrabou et al. 2009), whereas the 2011 event off the west coast of Australia has been recognized as being a driving factor in the regime shift there from temperate kelp forests to the beginnings of a coral reef system (Wernberg et al. 2013). Because the inquiry into MHWs is a relatively new endeavour none of these studies provided adequate definitions for what constitutes a MHW, and to that end Hobday et al. (2016) have defined it as ``a prolonged discrete anomalously warm water event that can be described by its duration, intensity, rate of evolution, and spatial extent''. By applying the MHW definition to the aforementioned events, Hobday et al. (2016) were able to derive statistical features of the MHWs, such as their frequency along a time series and maximum and cumulative intensity. Whereas extreme hot events may be demonstrably damaging to organisms and ecosystems, extreme cold events also have the potential to negatively impact organisms and ecosystems.

While MHWs are becoming reasonably well known by virtue of their increasing frequency and intensity, there is less information about the ecological effects of extreme cold events. Anomalous cold events are projected to become less frequent under future climatic scenarios, but there are also examples of them becoming more frequent in some small localities (Gershunov and Douville 2008; Matthes and Rinke 2015). Extreme cold events (here called ‘marine cold spells’, MCS) are frequently lethal (Woodward 1987) and are known to cause fish (Gunter 1940, 1951; Holt and Holt 1983) and invertebrate (Gunter 1951; Crisp 1964) kills, the death of juvenile and sub-adult manatees (O'Shea et al. 1985; Marsh et al. 1986) as well as affecting organismal physiological tolerances, life history strategies, and habitat requirements (Ellis 2015). Cold temperatures are therefore very important in setting species distribution limits, particularly limiting their range north- or southwards towards high latitudes (Firth et al. 2011), and the timing of the onset of the growing season (Jentsch et al. 2007). At an ecosystem level there is still a paucity of information on effects of MCSs, but it is easy to postulate how population-level consequences might aggregate to drive whole ecosystem responses (e.g. Kreyling et al. 2008; Rehage et al. 2016). Indeed, the range contractions of ecosystem engineer species such as mussels have been shown to relate to extreme cold events (e.g. Firth et al. 2011, 2015). Many of the extreme cold events that have had recorded negative impacts on individuals and ecosystems have been caused by atmospheric cold events, not by oceanographic phenomena (e.g. Gunter 1940; Firth et al. 2011). The question then is, in what way do MCSs, as defined here, affect ecosystems differently than routine upwelling? Is it the link to atmospheric forcing, or may a MCS capable of mass mortalities and ecosystem change be caused by the intensifications of coastal upwelling processes? Little research yet exists that investigates this question other than to link anoxia and other negative factors from problematic phytoplankton blooms caused by extreme upwelling events to create lethal conditions for species living within upwelling regions (e.g. Laboy-Nieves et al. 2001). Whereas anoxia is a problem attributable to phytoplankton blooms themselves (Diaz and Rosenberg 2008) and not the extreme cold temperatures \textit{per se}, if a relationship can be shown between MCSs and anoxia resulting from algal blooms it would provide extremely valuable insight into how coastal ecosystems respond to climatic change. To this end it serves as a constructive first step to define MCSs as the negative inverse of MHWs for the purposes of this investigation however.

Hobday et al. (2016) applied their MHW framework to \sfrac{1}{4}\degree~NOAA optimally interpolated sea surface temperature (hereafter referred to as OISST; Reynolds et al., 2007) data, but warned users to be cognisant that different data sets would provide different kinds of information pertaining to the heat waves. Our aims here were two-fold. Firstly, we applied the MHW (MCS) definition to datasets of \emph{in situ} and gridded SST temperature time series collected at different scales along the South African coast for the three different coastal sections, each variously forced by the Agulhas and Benguela Currents and regional aspects of the coastal bathymetry and geomorphology. These regional drivers of the thermal regime (east, south and west coast) coupled with local modifications (coastal vs. offshore) can be expected to impart different thermal signatures on the temperature data sets and manifest in differences in the metrics of MHWs (MCSs). Secondly, we aim to discuss the significance of MHWs (MCSs) within the context of the data sets’ inherent differences and the various dynamical properties that then emerge because of the regional oceanographic context, so as to provide a mechanistic understanding of the nature and origin of MHWs (MCSs) in three oceanographically distinct ocean/coastal regions.

To add a mechanistic understanding of the drivers of MHWs (MCSs) manifesting in the coastal environment, we hypothesised that coastal MHW (MCS) events could either be coupled with synoptic scale processes perturbing the offshore region at scales of 100s of km, or originate solely at a local scale as isolated incidents. Investigating the former possibility required the assessment of concurrent gridded SSTs derived from daily OISST data product, extracted for the bounding boxes in Figure 1, averaged spatially, and lagged or led by a number of days relative to the onset of the events at the coast. This analysis centres around the top three MHWs (MCSs) ranked with respect to cumulative intensity [\degree C~\texttimes~days] for each of 21 coastal sites. The rates of co-occurrence of coastal with mesoscale MHWs (MCSs) are used in part to understand how many of the extreme events detected in all three coastal sections originate at the coast or are artefacts of warming (cooling) in the respective currents. We think that this approach will yield considerable insight into the nature and variability of the thermal regime of nearshore seawater.

\section{Methods}
\subsection{Study region}\todo{I think this must be shortened significantly. I reads like a biologist wrote it --- which it did!}
The variety of oceanographic features around South Africa provide a natural testing bed for the potential effects of geographic forcing of oceanographic phenomena on the occurrence and frequency of MHWs and MCSs. The west coast of South Africa is dominated by the temperate Benguela Current, which is one of the four Eastern Boundary Upwelling System (EBUS) of the world (Hutchings et al. 2009). This area may experience large annual ranges in temperature and the many strong localised upwelling cells retard the more regular seasonal signal one would expect from ocean temperature. The Benguela Current does not regularly flow farther west than Cape Point before it meets the warmer Agulhas Current. The Agulhas Current flows in a south-westerly direction along the eastern shores of South Africa, which then retroflects back into the southern Indian Ocean (Hutchings et al. 2009). The south coast is dominated by a wide slab of continental shelf, the Agulhas Bank, jutting out south of South Africa, which plays host to the Agulhas Current as it widens out thus causing the Agulhas Current to slow down and cool off (Roberts 2004). This process is notoriously volatile and the south coast experiences the largest ranges in annual temperatures and variability of the three coasts. It has also been theorised that an upwelling cell exists along this coastline (Roberts 2004). There are many embayments on the south coast and it is thought that the thermal heating that occurs therein lends to the range and variability in temperatures seen on this coastal section. As varied as the south coast is, the east coast is stable. The continental shelf along the east coast is very narrow and the Agulhas current flows evenly southward toward the south coast. There are some small upwelling cells along this stretch of coastline caused by sheer forcing from the speed of the Agulhas current (Lutjeharms et al. 2003).

The sites selected for this study (see the section on \emph{Temperature data} below) represent the full thermal range and variability along the coast. Annual mean (SD) coastal seawater temperatures range from 12.3 (1.2) \degree C at the north western limit near the Namibian border (Site 1) to 24.4 (2.0) \degree C on the east coast near the Mozambican border (Site 21). The Agulhas and Benguela Currents modulate temperatures along this \emph{ca}. 2,700 km stretch of coastline. The southward flowing Agulhas Current has an overriding effect on the east coast of South Africa, and extends as far west as False Bay (Sites 5--21; Figure 1). This warm temperate region (Lüning 1990) occupies a continental shelf ranging in width from \emph{ca}. 4--200 km (Figure 1). Within this region, particularly around the towns of Port Alfred and Port Elizabeth (Sites 15-–17), topographically driven upwelling is sometimes present. The northward flowing Benguela Current is an Eastern Boundary Upwelling System (EBUS) maintained by prevailing south-easterly trade winds, which particularly influences the western side of the Cape Peninsula (Sites 2-–4) northwards to about 16\degree S. The influence of the Benguela Current here defines a cool temperate regime, with the range of monthly mean temperatures at most sections intermediate between cold temperate and warm temperate (Lüning, 1990).

The global latitudinal gradient of diminishing temperature with increasing latitude is only seen along the east coast (Figure 1), where the annual mean temperature decreases from 24.4 \degree C (Site 21) to 17.9 \degree C (Site 18). The alongshore thermal gradient for this 950 km stretch of coastline is \emph{ca}. 0.7 \degree C per 100 km, with steeper gradients near Sodwana. The latitudinal gradient largely reverses in direction along the west coast (Sites 1-–4, i.e. temperatures become slightly cooler further north. On average, these data indicate an increase in inshore annual mean temperatures from west to east (Sites 1-–21) of 12.3–-24.4 \degree C. In February the thermal range is 13.7 \degree C, while in August it is 10.5 \degree C. In August the west–east temperature transition is smooth whereas in February substantial warm fluctuations in the mean monthly temperature are observed in embayments such as Site 3, False Bay (Sites 5-–7) and many sites along the south coast (Sites 8–-17).

\subsection{Temperature data}
We use two sources of seawater temperature data. The first dataset is comprised of 127 South African \emph{in situ} coastal seawater temperature time series (Smit et al. 2013) derived from daily measurements up to 40 years in duration with a mean duration of \emph{ca}. 19 years. Whereas these \emph{in situ} time series are generally shorter than the recommended 30 year minimum (Hobday et al. 2016) and have some small amounts of missing data, it is our opinion that the benefit of using \emph{in situ} data over satellite data is that they give a better representation of the thermal characteristics near the coastline, a region where satellite SST measurements have been shown to perform poorly (e.g. Smale and Wernberg 2009; Castillo and Lima 2010). In a South African context, Smit et al. (2013) have shown that satellite SST data display a warm bias as large as 6\degree C over \emph{in situ} temperatures in the nearshore environment. In an attempt to compromise between the proscribed requirements in Hobday et al. (2016) of a 30 year minimum and no missing data, all time series under 10 years in length were eliminated. Next, our 127 time series were screened and those missing more than 10\% of their daily values were removed, leaving a total of 21 time series. Care was taken to select continuous series with as few as possible consecutive missing values, since having regions in the data with more than two consecutive missing data points interferes with the identification of the anomalous events (see below). These stations were classified into three coastal sections defined by properties of their oceanography and biogeography (Smit et al. 2013). The meta data for these time series and the coastal sections they were aggregated into may be found in Table 1 and the site localities are displayed spatially in Figure 1.

The second set of temperature data used in this study are the daily \sfrac{1}{4}\degree~NOAA optimally interpolated sea surface temperature (OISST; Reynolds et al. 2007) derived from the Advanced Very High Resolution Radiometer (AVHRR). To compare the OISST and \emph{in situ} time series, shore-normal transects were drawn from each of the 21 sites extending to the 200 m isobath. The OISST data were then extracted at each of the roughly 25~\texttimes~25 km pixels along these transects, shown as black boxes in Figure 1. Where the shelf was less than 25 km wide (Sites 17-–21) the nearest `ocean' pixel to the \emph{in situ} time series coordinates was used. The individual time series within each pixel were then averaged along each transect corresponding to the 21 \emph{in situ} sites. This produced 21 OISST time series that could then be analysed for MHWs (MCSs) in the same way as the \emph{in situ} data. Note that the OISST time series had valid data covering 1982--2014 which did not match exactly the coverage by individual \emph{in situ} sites.

\subsection{Defining and calculating MHWs and MCSs}
MHWs are ``discrete prolonged anomalously warm water events in a particular location.'' Here we introduce the opposite but analogous concept of a Marine Cold Spell (MCS), which is calculated in the same manner as a MHW, except that events are detected as deviations below a seasonally varying anomalously low threshold relative to the site’s climatology. Although MCS intensities are calculated as negative values (i.e. anomalies) they are reported here as absolute values.
A Python script (https://github.com/ecjoliver/marineHeatWaves; see Hobday et al. (2016)) was used to calculate the MHWs and MCSs for both the \emph{in situ} and OISST time series, producing the metrics in Table 1. The individual events detected and their attendant statistics were meaned into a series of annual values. These annual values were then meaned for each coastal section for later comparison.

To detect the individual events, a climatological mean and 90th and 10th percentiles were calculated for each day of the year by pooling all data within an 11-day window across all years. MHWs (MCSs) were detected as periods of time when temperatures exceeded the 90th (10th) percentile for at least five days. The implication is therefore that MHWs (MCSs) could develop in winter (summer) months. Since our \emph{in situ} time series are of differing lengths we calculated the climatology over all available years; in the case of the OISST data, climatologies were calculated over a 30-year base period (1982-–2012) Furthermore, the algorithm found discrete events with well-defined start and end dates, but `breaks' between events lasting $\leq$2 days followed by subsequent $\geq$5 day events were considered as continuous events. Once events were defined, a set of metrics were calculated including maximum and mean intensity (measured as anomalies relative to the climatological mean), duration (time between start and end dates), and cumulative intensity (the integrated intensity over the duration of the event, analogous to degree-heating-days).

Because MHWs (MCSs) are thus calculated by percentiles rather than maximum values, any time of year could be shown to be experiencing a MHW (MCS). This is an important consideration as unusually warm waters occurring during the winter months of a year, the time when many species need cold water for effective spawning spore release, can have a negative effect on the recruitment success of that population for the year (Wernberg et al. 20xx).

It is important to understand that MHWs can result from a combination of atmospheric forcing and oceanic processes, but that the approach here aims only to shed light on the oceanic drivers by virtue of the inclusion of mesoscale OISST data linked with the coastal \emph{in situ} data sets.

In order to better understand the potential impact mesoscale phenomena have on coastal events, the rates of co-occurrence between the MHWs (MCSs) found within each time series between the two datasets were compared. This was initially done by taking each event (warm and cold) within an \emph{in situ} time series and looking for an event occurring within the OISST time series at the same site within a certain period of time before the \emph{in situ} date. These co-occurrence proportions were then used to describe how often the mesoscale oceanography off the coast pre-empted the extreme events occurring along the coastline. All events occurring on dates not found in the matching time series were removed from this calculation. The sum of events found to occur within similar times was then divided by the total number of \emph{in situ} events checked against the OISST data to produce a co-occurrence proportion. The proportions of co-occurrence were then recalculated controlling for the amount of lag used when comparing the two different datasets for concurrent events, as well as the directionality used for this comparison. In other words, a range of lag [days] from 2--14 was used for each site to see how far apart events generally occurred and the lag period used was also applied only after the \emph{in situ} date, as well as both before and after the date, effectively doubling the range of the lag. This allowed us to see how often the \emph{in situ} event pre-empted the mesoscale event as well as seeing broadly the amounts of co-occurrence occurring between the two data sets.

Besides controlling for the length and direction of lag, the size of the events themselves (cumulative intensity [\degree C~\texttimes~days]) were compared. This was accomplished by controlling the pool of events with which to compare the datasets per site in steps of 10th percentiles. This progressively removed smaller events until only the larger events were being compared. This allowed us to track the co-occurrence of only the largest events, reducing the overall proportion of co-occurrence found within each site as caused by the large amount of smaller events occurring at similar times as other larger events.

The top three MHWs (MCSs) for each \emph{in situ} and OISST time series as defined by cumulative intensity [\degree C~\texttimes~days] were also noted in order to visually compare the co-occurrence of events in detail, both within and between the different datasets. Using the OISST data, images of the South African ocean temperature on the dates for the largest MHWs and MCSs (cumulative intensity [\degree C~\texttimes~days]) for the south and west coasts from the \emph{in situ} datasets were extracted and displayed on a map to show the spatial extent of any potentially co-occurring event in regions offshore from the coastline of South Africa.

Given that the anthropogenic forcing of climate change is predicted to increase the temperature of most of the ocean over time, it stands to reason that, as a function of the 90th and 10th percentiles, one would expect to see the larger MHWs near the end of the time series, and the larger MCSs near the beginning. \todo{Was this in fact done?} This can be tracked visually by looking at the top three warm and cold events for each time series.

\section{Results}

Here are two sample references: \cite{Feynman1963118,Dirac1953888}.

\section*{References}

\bibliography{mybibfile}

\end{document}