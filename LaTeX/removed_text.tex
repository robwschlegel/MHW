Little research yet exists that investigates this question other than to link anoxia and other negative factors from problematic phytoplankton blooms caused by extreme upwelling events to create lethal conditions for species living within upwelling regions \citep{Laboy-nieves2001}. Whereas anoxia is a problem attributable to phytoplankton blooms themselves \citep{Diaz2008} and not the extreme cold temperatures \textit{per se}, if a relationship can be shown between MCSs and anoxia resulting from algal blooms it would provide extremely valuable insight into how coastal ecosystems respond to climatic change.

Investigating the possibility of synoptic scale process driving coastal events required the assessment of concurrent gridded SSTs derived from daily OISST data product, extracted for the bounding boxes seen in \Cref{fig:Figure1}, averaged spatially, and lagged or led by a number of days relative to the onset of the events at the coast. This analysis centres around the top three MHWs (MCSs) ranked with respect to cumulative intensity for each of 21 coastal sites. The rates of co-occurrence of coastal with mesoscale MHWs (MCSs) are used in part to understand how many of the extreme events detected in all three coastal sections originate at the coast or are artefacts of warming (cooling) in the respective currents. We think that this approach will yield considerable insight into the nature and variability of the thermal regime of nearshore seawater.
